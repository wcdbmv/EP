\documentclass[a4paper, 14pt]{extreport}

\include{preamble}

\begin{document}

\include{00-title-page}

\begin{MainPart}

\section{Задание}

Реализация ПФЭ на имитационной модели функционирования СМО.

Составить матрицу планирования для проведения ПФЭ для одноканальной СМО с одним генератором заявок.

Интервалы варьирования факторов выбрать на основе результатов первой л. р., в рамках которой исследовались зависимости выходной величины (среднего времени ожидания (пребывания)) от входных параметров (интенсивность поступления, интенсивность обслуживания).
В итоге получить зависимость выходной величины от загрузки.

По результатам ПФЭ вычислить коэффициенты линейной и частично нелинейной регрессионной зависимости.

Предусмотреть возможность сравнения рассчитанной величины с реальной, полученной по результатам имитационного моделирования.

\section{Теоретическая часть}

Коэффициент загрузки одноканальной СМО и среднее время ожидания определяются формулами:
\begin{equation}
	\label{eqn:load-and-wait}
	\rho = \frac\lambda\mu, \qquad \overline{t_{\text{ож}}} = \frac{\rho}{(1 - \rho)\lambda}
\end{equation}
где $\lambda$ — интенсивность входящего потока заявок, $\mu$ — интенсивность обслуживания.

Интервалы времени между приходом заявок распределены по равномерному закону $(X \sim R(a, b))$, коэффициенты $a$ и $b$ которого рассчитываются как
\begin{equation}
	\begin{aligned}
		a &= \frac1\lambda - \sqrt{3}\sigma_\lambda, \\
		b &= \frac1\lambda + \sqrt{3}\sigma_\lambda.
	\end{aligned}
\end{equation}

Времена обслуживания заявок распределены по закону Вейбулла $(X \sim W(k, \lambda))$ c параметром $k = 2$.
Коэффициент $\lambda$ распределения определяется по формуле
\begin{equation}
	\begin{aligned}
		\lambda &= \frac{1}{\mu\Gamma{\left(1 + \frac1k\right)}}.
	\end{aligned}
\end{equation}

Для проведения ПФЭ при трёх факторах необходимо $N = 2^3$ опытов.
В таблице \ref{tbl:planning-matrix} представлена матрица планирования для проведения ПФЭ.

\begin{table}[ht]
	\centering
	\captionsetup{justification=centering}
	\caption{Матрица планирования}
	\label{tbl:planning-matrix}
	\begin{tabular}{|c|c|c|c|c|c|c|c|c|}
		\hline
		№ опыта & $x_1$ & $x_2$ & $x_3$ & $x_1x_2$ & $x_1x_3$ & $x_2x_3$ & $x_1x_2x_3$ & $y$   \\ \hline
		$1$     & $-1$  & $-1$  & $-1$  & $+1$     & $+1$     & $+1$     & $-1$        & $y_1$ \\ \hline
		$2$     & $-1$  & $-1$  & $+1$  & $+1$     & $-1$     & $-1$     & $+1$        & $y_2$ \\ \hline
		$3$     & $-1$  & $+1$  & $-1$  & $-1$     & $+1$     & $-1$     & $+1$        & $y_3$ \\ \hline
		$4$     & $-1$  & $+1$  & $+1$  & $-1$     & $-1$     & $+1$     & $-1$        & $y_4$ \\ \hline
		$5$     & $+1$  & $-1$  & $-1$  & $-1$     & $-1$     & $+1$     & $+1$        & $y_5$ \\ \hline
		$6$     & $+1$  & $-1$  & $+1$  & $-1$     & $+1$     & $-1$     & $-1$        & $y_6$ \\ \hline
		$7$     & $+1$  & $+1$  & $-1$  & $+1$     & $-1$     & $-1$     & $-1$        & $y_7$ \\ \hline
		$8$     & $+1$  & $+1$  & $+1$  & $+1$     & $+1$     & $+1$     & $+1$        & $y_8$ \\ \hline
	\end{tabular}
\end{table}

Линейная регрессия для трёх факторов:
\begin{equation}
	y = a_0 + a_1x_1 + a_2x_2 + a_3x_3.
\end{equation}

Частично нелинейная регрессия для трёх факторов:
\begin{equation}
	y = a_0 + a_1x_1 + a_2x_2 + a_3x_3 + a_{12}x_1x_2 + a_{13}x_1x_3 + a_{23}x_2x_3 + a_{123}x_1x_2x_3.
\end{equation}

Однородность ряда дисперсий можно проверить по критерию Кохрена.
При равномерном дублировании экспериментов однородность ряда дисперсий проверяют с помощью $G$-критерия Кохрена, представляющего собой отношение максимальной дисперсии к сумме всех дисперсий.
Дисперсии однородны, если расчётное значение $G_p$-критерия не превышает табличного значения.
Иначе исследуемая величина неоднородна и не подчиняется нормальному закону.

Если ряд дисперсий однороден, то дисперсию воспроизводимости вычисляют, как отношение суммы ряда дисперсий к числу строк матрицы планирования.

Дисперсией воспроизводимости эксперимента $D_y$ называется дисперсия наблюдаемой переменной.
Эксперимент идеален при $D_y = 0$.

После расчёта коэффициентов модели и проверки их значимости определяют дисперсию $s^2_{\text{ад}}$ адекватности.
Остаточная дисперсия, или дисперсия адекватности, характеризует рассеяние эмпирических значений $y$ относительно расчётных $\overline y$, определённых по найденному уравнению регрессии.

Проверку гипотезы адекватности найденной модели производят по $F$-критерию Фишера, рассчитываемому как отношение дисперсии адекватности к дисперсии воспроизводимости.

\section{Реализация}

На рисунке \ref{img:interface} представлен интерфейс программы.

\img{width=\linewidth}{interface}{Интерфейс программы}

\section{Моделирование}

Результат работы программы представлен на рисунке \ref{img:simulation}.

\img{width=\linewidth}{simulation}{Результат работы программы}

% \section{Выводы} % это лишнее)

\end{MainPart}
\end{document}
