\documentclass[a4paper, 14pt]{extreport}

\include{preamble}

\begin{document}

\include{00-title-page}

\begin{MainPart}

\section{Задание}

Реализация ПФЭ и ДФЭ на имитационной модели функционирования СМО.

Составить матрицу планирования для проведения ПФЭ для СМО с двумя генератором заявок (в исходную СМО добавить второй генератор).

Интервалы варьирования факторов выбрать на основе результатов первой л. р., в рамках которой исследовались зависимости выходной величины (среднего времени ожидания (пребывания)) от входных параметров (интенсивность поступления, интенсивность обслуживания).
В итоге получить зависимость выходной величины от загрузки.

По результатам ПФЭ вычислить коэффициенты линейной и частично нелинейной регрессионной зависимости.

Составить матрицу планирования ДФЭ.
Провести ДФЭ.
Рассчитать коэффициенты линейной и частично нелинейной регрессионной зависимости.

Предусмотреть возможность сравнения рассчитанной величины с реальной, полученной по результатам имитационного моделирования.


\section{Теоретическая часть}

Коэффициент загрузки одноканальной СМО и среднее время ожидания определяются формулами:
\begin{equation}
	\label{eqn:load-and-wait}
	\rho = \frac\lambda\mu, \qquad \overline{t_{\text{ож}}} = \frac{\rho}{(1 - \rho)\lambda}
\end{equation}
где $\lambda$ — интенсивность входящего потока заявок, $\mu$ — интенсивность обслуживания.

Интервалы времени между приходом заявок распределены по равномерному закону $(X \sim R(a, b))$, коэффициенты $a$ и $b$ которого рассчитываются как
\begin{equation}
	\begin{aligned}
		a &= \frac1\lambda - \sqrt{3}\sigma_\lambda, \\
		b &= \frac1\lambda + \sqrt{3}\sigma_\lambda.
	\end{aligned}
\end{equation}

Времена обслуживания заявок распределены по закону Вейбулла $(X \sim W(k, \lambda))$ c параметром $k = 2$.
Коэффициент $\lambda$ распределения определяется по формуле
\begin{equation}
	\begin{aligned}
		\lambda &= \frac{1}{\mu\Gamma{\left(1 + \frac1k\right)}}.
	\end{aligned}
\end{equation}

Для проведения ПФЭ для $n$ факторов необходимо $N = 2^n$ опытов.
В таблице \ref{tbl:full-planning-matrix} представлена матрица планирования для проведения ПФЭ.

\begin{table}[hbt!]
	\centering
	\captionsetup{justification=centering}
	\caption{Матрица планирования для проведения ПФЭ}
	\label{tbl:full-planning-matrix}
	\begin{tabular}{|c|c|c|c|c|c|c|}
		\hline
		№ опыта & $x_1$ & $x_2$ & $x_3$ & $x_4$ & $x_5$ & $y$      \\ \hline
		$1$     & $-1$  & $-1$  & $-1$  & $-1$  & $-1$  & $y_1$    \\ \hline
		$2$     & $+1$  & $-1$  & $-1$  & $-1$  & $-1$  & $y_2$    \\ \hline
		$3$     & $-1$  & $+1$  & $-1$  & $-1$  & $-1$  & $y_3$    \\ \hline
		$4$     & $+1$  & $+1$  & $-1$  & $-1$  & $-1$  & $y_4$    \\ \hline
		$5$     & $-1$  & $-1$  & $+1$  & $-1$  & $-1$  & $y_5$    \\ \hline
		$6$     & $+1$  & $-1$  & $+1$  & $-1$  & $-1$  & $y_6$    \\ \hline
		$7$     & $-1$  & $+1$  & $+1$  & $-1$  & $-1$  & $y_7$    \\ \hline
		$8$     & $+1$  & $+1$  & $+1$  & $-1$  & $-1$  & $y_8$    \\ \hline
		$9$     & $-1$  & $-1$  & $-1$  & $+1$  & $-1$  & $y_9$    \\ \hline
		$10$    & $+1$  & $-1$  & $-1$  & $+1$  & $-1$  & $y_{10}$ \\ \hline
		$11$    & $-1$  & $+1$  & $-1$  & $+1$  & $-1$  & $y_{11}$ \\ \hline
		$12$    & $+1$  & $+1$  & $-1$  & $+1$  & $-1$  & $y_{12}$ \\ \hline
		$13$    & $-1$  & $-1$  & $+1$  & $+1$  & $-1$  & $y_{13}$ \\ \hline
		$14$    & $+1$  & $-1$  & $+1$  & $+1$  & $-1$  & $y_{14}$ \\ \hline
		$15$    & $-1$  & $+1$  & $+1$  & $+1$  & $-1$  & $y_{15}$ \\ \hline
		$16$    & $+1$  & $+1$  & $+1$  & $+1$  & $-1$  & $y_{16}$ \\ \hline
		$17$    & $-1$  & $-1$  & $-1$  & $-1$  & $+1$  & $y_{17}$ \\ \hline
		$18$    & $+1$  & $-1$  & $-1$  & $-1$  & $+1$  & $y_{18}$ \\ \hline
		$19$    & $-1$  & $+1$  & $-1$  & $-1$  & $+1$  & $y_{19}$ \\ \hline
		$20$    & $+1$  & $+1$  & $-1$  & $-1$  & $+1$  & $y_{20}$ \\ \hline
		$21$    & $-1$  & $-1$  & $+1$  & $-1$  & $+1$  & $y_{21}$ \\ \hline
		$22$    & $+1$  & $-1$  & $+1$  & $-1$  & $+1$  & $y_{22}$ \\ \hline
		$23$    & $-1$  & $+1$  & $+1$  & $-1$  & $+1$  & $y_{23}$ \\ \hline
		$24$    & $+1$  & $+1$  & $+1$  & $-1$  & $+1$  & $y_{24}$ \\ \hline
		$25$    & $-1$  & $-1$  & $-1$  & $+1$  & $+1$  & $y_{25}$ \\ \hline
		$26$    & $+1$  & $-1$  & $-1$  & $+1$  & $+1$  & $y_{26}$ \\ \hline
		$27$    & $-1$  & $+1$  & $-1$  & $+1$  & $+1$  & $y_{27}$ \\ \hline
		$28$    & $+1$  & $+1$  & $-1$  & $+1$  & $+1$  & $y_{28}$ \\ \hline
		$29$    & $-1$  & $-1$  & $+1$  & $+1$  & $+1$  & $y_{29}$ \\ \hline
		$30$    & $+1$  & $-1$  & $+1$  & $+1$  & $+1$  & $y_{30}$ \\ \hline
		$31$    & $-1$  & $+1$  & $+1$  & $+1$  & $+1$  & $y_{31}$ \\ \hline
		$32$    & $+1$  & $+1$  & $+1$  & $+1$  & $+1$  & $y_{32}$ \\ \hline
	\end{tabular}
\end{table}

В нашем случае $n = 5$ факторов:
\begin{enumerate}
	\item $x_1$ — интенсивность $\lambda_1$ поступления заявок генератора 1,
	\item $x_2$ — интенсивность $\lambda_2$ поступления заявок генератора 2,
	\item $x_3$ — интенсивность $\mu$ времён обслуживания ОА,
	\item $x_4$ — СКО $\sigma(\lambda_1)$ поступления заявок генератора 1,
	\item $x_5$ — СКО $\sigma(\lambda_2)$ поступления заявок генератора 2.
\end{enumerate}

Для проведения ДФЭ для $n$ факторов необходимо $N = 2^{n - k}$ опытов, где $k$ — показатель дробности плана.
В таблице \ref{tbl:frac-planning-matrix} представлена матрица планирования для проведения ПФЭ.

\begin{table}[hbt!]
	\centering
	\captionsetup{justification=centering}
	\caption{Матрица планирования для проведения ДФЭ}
	\label{tbl:frac-planning-matrix}
	\begin{tabular}{|c|c|c|c|c|c|c|}
		\hline
		№ опыта & $x_1$ & $x_2$ & $x_3$ & $x_4$ & $x_5$ & $y$      \\ \hline
		$1$     & $-1$  & $-1$  & $-1$  & $+1$  & $+1$  & $y_1$    \\ \hline
		$2$     & $+1$  & $-1$  & $-1$  & $-1$  & $+1$  & $y_2$    \\ \hline
		$3$     & $-1$  & $+1$  & $-1$  & $+1$  & $-1$  & $y_3$    \\ \hline
		$4$     & $+1$  & $+1$  & $-1$  & $-1$  & $-1$  & $y_4$    \\ \hline
		$5$     & $-1$  & $-1$  & $+1$  & $-1$  & $-1$  & $y_5$    \\ \hline
		$6$     & $+1$  & $-1$  & $+1$  & $+1$  & $-1$  & $y_6$    \\ \hline
		$7$     & $-1$  & $+1$  & $+1$  & $-1$  & $+1$  & $y_7$    \\ \hline
		$8$     & $+1$  & $+1$  & $+1$  & $+1$  & $+1$  & $y_8$    \\ \hline
	\end{tabular}
\end{table}

В нашем случае план ДФЭ является четвертьрепликой: первые три фактора варьируем как ранее, а генерирующие соотношения для остальных:
\begin{equation}
	x_4 = x_1x_3, \qquad x_5 = x_2x_3.
\end{equation}

Определяющий контраст:
\begin{equation}
	x_4^2 = x_1x_3x_4 = 1, \qquad x_5^2 = x_2x_3x_5 = 1.
\end{equation}

Система совместных оценок:
\begin{gather*}
	b_1 = \beta_1 + \beta_{34}, \quad
	b_2 = \beta_2 + \beta_{35}, \quad
	b_3 = \beta_3 + \beta_{14} = \beta_3 + \beta_{25} \\
	b_4 = \beta_4 + \beta_{13}, \quad
	b_5 = \beta_5 + \beta_{23}.
\end{gather*}

Линейная и частично-нелинейная регрессии для $n$ факторов:
\begin{equation}
	\hat y = b_0 + \sum_{i = 1}^{n}b_ix_i; \quad \hat u = b_0 + \sum_{1 \leqslant k \leqslant n}\sum_{1 \leqslant i_1 < \ldots < i_k \leqslant n}b_{i_1\ldots i_k}\cdot x_{i_1} \cdot \ldots \cdot x_{i_k}
\end{equation}

\section{Реализация}

На рисунке \ref{img:interface} представлен интерфейс программы.

\img{width=\linewidth}{interface}{Интерфейс программы}

\section{Моделирование}

Результат работы программы представлен на рисунке \ref{img:simulation}.

\img{width=\linewidth}{simulation}{Результат работы программы}

% \section{Выводы} % это лишнее)

\end{MainPart}
\end{document}
