\documentclass[a4paper, 14pt]{extreport}

\include{preamble}

\begin{document}

\include{00-title-page}

\begin{MainPart}

\section{Задание}

Составить матрицу планирования для проведения ОЦКП для СМО с двумя генератором заявок (в исходную СМО добавить второй генератор).

Интервалы варьирования факторов выбрать на основе результатов первой л. р., в рамках которой исследовались зависимости выходной величины (среднего времени ожидания (пребывания)) от входных параметров (интенсивность поступления, интенсивность обслуживания).
В итоге получить зависимость выходной величифны от загрузки.

Для ОЦКП рассчитать необходимые величины (звёздное плечо).

По результатам ОЦКП вычислить коэффициенты нелинейной регрессионной зависимости.

Предусмотреть возможность сравнения рассчитанной величины с реальной, полученной по результатам имитационного моделирования.

\section{Теоретическая часть}

Коэффициент загрузки одноканальной СМО и среднее время ожидания определяются формулами:
\begin{equation}
	\label{eqn:load-and-wait}
	\rho = \frac\lambda\mu, \qquad \overline{t_{\text{ож}}} = \frac{\rho}{(1 - \rho)\lambda}
\end{equation}
где $\lambda$ — интенсивность входящего потока заявок, $\mu$ — интенсивность обслуживания.

Интервалы времени между приходом заявок распределены по равномерному закону $(X \sim R(a, b))$, коэффициенты $a$ и $b$ которого рассчитываются как
\begin{equation}
	\begin{aligned}
		a &= \frac1\lambda - \sqrt{3}\sigma_\lambda, \\
		b &= \frac1\lambda + \sqrt{3}\sigma_\lambda.
	\end{aligned}
\end{equation}

Времена обслуживания заявок распределены по закону Вейбулла $(X \sim W(k, \lambda))$ c параметром $k = 2$.
Коэффициент $\lambda$ распределения определяется по формуле
\begin{equation}
	\begin{aligned}
		\lambda &= \frac{1}{\mu\Gamma{\left(1 + \frac1k\right)}}.
	\end{aligned}
\end{equation}

В нашей модели $n = 6$ факторов:
\begin{enumerate}
	\item $x_1$ — интенсивность $\lambda_1$ поступления заявок генератора 1,
	\item $x_2$ — интенсивность $\lambda_2$ поступления заявок генератора 2,
	\item $x_3$ — интенсивность $\mu_1$ времён обслуживания ОА заявок 1-го типа,
	\item $x_4$ — интенсивность $\mu_2$ времён обслуживания ОА заявок 2-го типа,
	\item $x_5$ — СКО $\sigma(\lambda_1)$ поступления заявок генератора 1,
	\item $x_6$ — СКО $\sigma(\lambda_2)$ поступления заявок генератора 2.
\end{enumerate}

В ОЦКП входят: ядро — план ПФЭ с $N_0 = 2^n$ точками плана, центральная точка плана ($n_c$) и по две «звёздные» точки для каждого фактора $n_\alpha = 2n$.
Общее число опытов в ОЦКП:
\begin{equation}
	N = 2^n + 2n + n_c = 2^6 + 2 \cdot 6 + 1 = 77.
\end{equation}

Нелинейная регрессия для $n$ факторов с учётом только двойного взаимодействия для обеспечения ортогонального свойства матрицы планирования эксперимента содержит некоторую постоянную $S$:
\begin{equation}
	\hat y = a_0 + \sum_{1 \leqslant i \leqslant n}a_ix_i + \sum_{1 \leqslant i < j \leqslant n}a_{ij}x_ix_j + \sum_{1 \leqslant i \leqslant n}a_{ii}(x_i^2-S).
\end{equation}

Из условия ортогональности матрицы планирования коэффициент $S$ и звёздное плечо $\alpha$ находятся как
\begin{gather}
	S = \sqrt{\frac {N_0}N} = 0.9116846116771, \\
	\alpha = \sqrt{\frac {N_0}2 \left(\sqrt{\frac N{N_0}} - 1\right)} = \sqrt{\frac {N_0}2 \left(\frac1S - 1\right)} = 1.760641232497.
\end{gather}

На рисунке \ref{img:matrix} представлена матрица планирования ОЦКП.

\img{width=0.95\linewidth}{matrix}{Матрица планирования ОЦКП}

\section{Реализация и моделирование}

\img{width=\linewidth}{interface}{Интерфейс программы}

% \section{Выводы} % это лишнее)

\end{MainPart}
\end{document}
