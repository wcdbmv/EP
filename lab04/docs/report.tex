\documentclass[a4paper, 14pt]{extreport}

\include{preamble}

\begin{document}

\include{00-title-page}

\begin{MainPart}

\section{Задание}

Составить матрицу планирования для проведения ОЦКП для СМО с двумя генератором заявок (в исходную СМО добавить второй генератор).

Интервалы варьирования факторов выбрать на основе результатов первой л. р., в рамках которой исследовались зависимости выходной величины (среднего времени ожидания (пребывания)) от входных параметров (интенсивность поступления, интенсивность обслуживания).
В итоге получить зависимость выходной величифны от загрузки.

Для ОЦКП рассчитать необходимые величины (звёздное плечо).

По результатам ОЦКП вычислить коэффициенты нелинейной регрессионной зависимости.

Предусмотреть возможность сравнения рассчитанной величины с реальной, полученной по результатам имитационного моделирования.

\section{Теоретическая часть}

Коэффициент загрузки одноканальной СМО и среднее время ожидания определяются формулами:
\begin{equation}
	\label{eqn:load-and-wait}
	\rho = \frac\lambda\mu, \qquad \overline{t_{\text{ож}}} = \frac{\rho}{(1 - \rho)\lambda}
\end{equation}
где $\lambda$ — интенсивность входящего потока заявок, $\mu$ — интенсивность обслуживания.

Интервалы времени между приходом заявок распределены по равномерному закону $(X \sim R(a, b))$, коэффициенты $a$ и $b$ которого рассчитываются как
\begin{equation}
	\begin{aligned}
		a &= \frac1\lambda - \sqrt{3}\sigma_\lambda, \\
		b &= \frac1\lambda + \sqrt{3}\sigma_\lambda.
	\end{aligned}
\end{equation}

Времена обслуживания заявок распределены по закону Вейбулла $(X \sim W(k, \lambda))$ c параметром $k = 2$.
Коэффициент $\lambda$ распределения определяется по формуле
\begin{equation}
	\begin{aligned}
		\lambda &= \frac{1}{\mu\Gamma{\left(1 + \frac1k\right)}}.
	\end{aligned}
\end{equation}

Для проведения ПФЭ для $n$ факторов необходимо $N = 2^n$ опытов.

В нашем случае $n = 6$ факторов:
\begin{enumerate}
	\item $x_1$ — интенсивность $\lambda_1$ поступления заявок генератора 1,
	\item $x_2$ — интенсивность $\lambda_2$ поступления заявок генератора 2,
	\item $x_3$ — интенсивность $\mu_1$ времён обслуживания ОА заявок 1-го типа,
	\item $x_4$ — интенсивность $\mu_2$ времён обслуживания ОА заявок 2-го типа,
	\item $x_5$ — СКО $\sigma(\lambda_1)$ поступления заявок генератора 1,
	\item $x_6$ — СКО $\sigma(\lambda_2)$ поступления заявок генератора 2.
\end{enumerate}

Нелинейная регрессия для $n$ факторов:
\begin{equation}
	\hat y = a_0 + \sum_{1 \leqslant k \leqslant n}\sum_{1 \leqslant i_1 \leqslant \ldots \leqslant i_k \leqslant n}a_{i_1\ldots i_k}\cdot x_{i_1} \cdot \ldots \cdot x_{i_k}
\end{equation}

\section{Реализация и моделирование}

\img{width=0.915\linewidth}{interface}{Интерфейс программы}

\img{width=0.915\linewidth}{simulation}{Результат работы программы}

% \section{Выводы} % это лишнее)

\end{MainPart}
\end{document}
