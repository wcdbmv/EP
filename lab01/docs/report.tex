\documentclass[a4paper, 14pt]{extreport}

\include{preamble}

\begin{document}

\include{00-title-page}

\begin{MainPart}

\section{Задание}

Разработать имитационную модель функционирования СМО.

СМО представляет собой одноканальную разомкнутую систему (один генератор заявок и один обслуживающий аппарат). Буфер имеет бесконечную емкость.

Закон поступления (генерации заявок) и закон распределения времени обслуживания заявок задается в таблице \ref{tbl:distributions} и выбирается в соответствии с номером в списке группы.

\begin{table}[H]
	\centering
	\captionsetup{justification=centering}
	\caption{Законы распределений}
	\label{tbl:distributions}
	\begin{tabular}{|l|l|l|}
		\hline
		\multicolumn{1}{|c|}{\textbf{№ варианта}} & \multicolumn{1}{c|}{\textbf{Первый закон}} & \multicolumn{1}{c|}{\textbf{Второй закон}} \\ \hline
		4                                         & Равномерный                                & Вейбулла                                   \\ \hline
	\end{tabular}
\end{table}

В качестве исходных данных пользователь задает интенсивность поступления заявок и интенсивность обслуживания заявок.
Программа должна выводить расчетную загрузку системы и фактическую, полученную по результатам моделирования.
Пользователь должен иметь возможность задавать время моделирования.

Если параметры законов распределения отличны от интенсивности, то предусмотреть ввод интенсивностей с дальнейшим пересчетом в программе этих величин в параметры закона.
В случае двухпараметрических законов пользователь задает интенсивность и ее разброс (среднеквадратическое отклонение).

Построить график зависимости выходного параметра (ср. время ожидания (пребывания) в зависимости от загрузки системы).

Предусмотреть наращивание системы путем добавления новых генераторов и обслуживающих аппаратов.

\section{Теоретическая часть}

Коэффициент загрузки одноканальной СМО и среднее время ожидания определяются формулами:
\begin{equation}
	\label{eqn:load-and-wait}
	\rho = \frac\lambda\mu, \qquad \overline{t_{\text{ож}}} = \frac{\rho}{(1 - \rho)\lambda}
\end{equation}
где $\lambda$ — интенсивность входящего потока заявок, $\mu$ — интенсивность обслуживания.

Интервалы времени между приходом заявок распределены по равномерному закону $(X \sim R(a, b))$:
\begin{equation}
	f_X(x) = \left\{
	\begin{aligned}
		&\frac{1}{b - a}, & x \in [a, b] \\
		&0,               & x \not\in [a, b]
	\end{aligned}
	\right..
\end{equation}

Математическое ожидание и дисперсия равномерного распределения:
\begin{equation}
	\mu_R = \frac{a + b}{2}, \qquad \sigma_R^2 = \frac{(b - a)^2}{12},
\end{equation}
отсюда
\begin{equation}
	\begin{aligned}
		a &= \mu_R - \sqrt{3}\sigma_R, \\
		b &= \mu_R + \sqrt{3}\sigma_R.
	\end{aligned}
\end{equation}

Времена обслуживания заявок распределены по закону Вейбулла $(X \sim W(k, \lambda))$:
\begin{equation}
	f_X(x) = \left\{
	\begin{aligned}
		& \frac k\lambda \left(\frac x\lambda\right)^{k-1} e^{-\left(\frac x\lambda\right)^k}, & x \geqslant 0 \\
		& 0,                                                                                   & x < 0
	\end{aligned}
	\right..
\end{equation}

Для избежания путаницы, в дальнейшем интенсивности будут помечаться штрихом: $\lambda'$, $\mu'$.

Математическое ожидание и дисперсия распределения Вейбуллы:
\begin{equation}
	\label{eqn:weibull-mean-and-variance}
	\mu_W = \lambda\Gamma{\left(1 + \frac1k\right)}, \qquad \sigma_W^2 = \lambda^2\Gamma{\left(1 + \frac2k\right)} - \mu_W^2.
\end{equation}

Коэффициенты распределения Вейбулла не могут быть выражены в элементарных функциях из формулы \eqref{eqn:weibull-mean-and-variance}.
Однако можно воспользоваться аппроксимацией:
\begin{equation}
	\begin{aligned}
		k       &= \left(\frac{\sigma_W}{\mu_W}\right)^{-1,086}, \\
		\lambda &= \frac{\mu_W}{\Gamma{\left(1 + \frac1k\right)}}.
	\end{aligned}
\end{equation}

Интенсивность — величина, обратная математическому ожиданию:
\begin{equation}
	\begin{aligned}
		\lambda' &= \frac{1}{\mu_R}, \\
		\mu'     &= \frac{1}{\mu_W}.
	\end{aligned}
\end{equation}

\section{Реализация}

На рисунке \ref{img:interface} представлен интерфейс программы.

\img{width=\linewidth}{interface}{Интерфейс программы}

\section{Моделирование}

Результат работы программы представлен на рисунке \ref{img:simulation}.

\img{width=\linewidth}{simulation}{Результат работы программы}

\section{Выводы}

В ходе выполнения лабораторной работы была разработана и программно реализована имитационная модель функционирования одноканальной СМО.

Результаты моделирования подтверждают теоретические расчёты: фактическая загрузка системы и среднее время ожидания приближаются к своим аналитическим формулам \eqref{eqn:load-and-wait}.
При стремлении коэффициента загрузки к единице, среднее время ожидания резко возрастает и стремится к бесконечности.

\end{MainPart}
\end{document}
